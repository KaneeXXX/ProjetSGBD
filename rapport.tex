\documentclass[12pt, a4paper]{article}
\usepackage[french]{babel}
\usepackage{parskip}
\usepackage[utf8]{inputenc} % Set input encoding to Unicode UTF-8
\usepackage[T1]{fontenc} % Set character encoding to T1/Cork
\usepackage{minted}
\usepackage{svg}
\usepackage{hyphenat}
\usepackage{amssymb}
\usepackage{tabularx}
\usepackage{booktabs}
\usepackage{adjustbox}
\usepackage{csquotes}
\usepackage{float}
\usepackage{enumitem}

\AtBeginDocument{\def\labelitemi{$\vcenter{\hbox{\small$\bullet$}}$}}
\setlist[itemize]{after=\newline}
\newcommand\att[1]{\textnhtt{#1}}

\title{Rapport du Projet de Bases de Données}
\author{Pradana \textsc{Aumars}
  \and
  David \textsc{Behrens}
  \and
  El Hadji Abdoul Aziz \textsc{Kane}
  \and
  Mohamed Lemine \textsc{Mohamed Ahmed}
  \and
  Mohamedou (Khalil) \textsc{Chrif M'hamed}}

\begin{document}
\maketitle
\section{Introduction}
\section{Analyse du problème}
\subsection*{Propriétés}
% Refuge
\att{emailRef}, % email du refuge
\att{nomRef}, % nom du refuge
\att{numTel}, % numéro de téléphone (optionnel) du refuge
\att{secGeo}, % secteur géographique du refuge
\att{dateOuv}, % date d'ouverture du refuge
\att{dateFerm}, % date de fermeture du refuge
\att{nbrPlacesRepas}, % nombre de places pour les repas dans un refuge
\att{nbrPlacesDormir}, % nombre de places pour dormir dans un refuge
\att{textePrez}, % texte de présentation d'un refuge
\att{typePayement}, % type de paiement pour un refuge
\att{prixNuite}, % prix d'une nuite dans un refuge
% Repas
\att{typeRepas}, % type de repas
\att{prixRepas}, % prix d'un repas
% Formation
\att{idForm}, % id de la formation
\att{anneeForm}, % annee de dispension de la formation
\att{nomForm}, % nom de la formation
\att{dateDemForm}, % date de démarrage de la formation
\att{dureeForm}, % durée en jours de la formation
\att{nbrPlacesForm}, % nombre de place total de la formation
\att{descriptForm}, % description de la formation
\att{prixForm}, % prix de la formation
\att{nbrActivite}, % nombre d'activités dans la formation
% Activité
\att{nomActivite}, % nom de l'activité
% Lot
\att{marque}, % marque du matériel
\att{modele}, % modèle du matériel
\att{anneeAchat}, % annee d'achat du matériel
\att{nbPiecesLot}, % nombre de pièces dans le lot
\att{prixPertDegr}, % prix en cas de perte/dégradation d'une pièce du lot
\att{texteInfo}, % texte (optionnel) sur le matériel
\att{anneePerem}, % année de péremption du matériel
% LocationMateriel
\att{idLocationMat}, % id de la location de matériel
\att{nbPiecesReserv}, % nombre de pièces réservées dans un lot
\att{dateRecup},
\att{dateRetour},
\att{nbPiecesPerdues},
% ArbreCategorie
\att{categorie}, % un noeud fils catégorie
\att{categoriePere}, % un noeud père catégorie
% Compte Utilisateur
\att{idUser}, % id du compte utilisateur
\att{coutReserv},
\att{sommeMaterielAbime},
\att{sommeRembourse},
% Membre
\att{emailMem}, % adresse mail du membre
\att{pwdMem}, % mot de passe du membre
\att{nomMem}, % nom du membre
\att{prenomMem}, % prénom du membre
\att{adrPostMem}, % adresse postale du membre
% Adhérent
\att{idAdherent}, % id de l'adhérent
% Réservation de refuge
\att{idReservRef}, % id d'une réservation de refuge
\att{dateReservRef}, % date d'une réservatinon d'une refuge
\att{heureReservRef}, % heure d'une réservation d'une refuge
\att{nbrNuitsReserv}, % nombre de nuits réservées
\att{nbrRepasReserv}, % nombre de repas réservés
\att{prixTotalReserv}, % prix total de la réservation
\att{nbRepasReserve},
% Réservation de formation
\att{idReservForm}, % id d'une réservation de formation
% NB : 0/1..* -- 0/1..* => attribut d'association / 
% Liste d'attente
\att{rangReservForm} % rang de l'adhérent dans la liste d'attente

\subsection*{Contraintes}
\begin{table}[H]
  \tiny
  \begin{adjustbox}{center}
\begin{tabularx}{0.98\paperwidth}{>{\raggedright}XXXX}
\textbf{Dépendances fonctionelles}
& \textbf{Contraintes de valeur}
& \textbf{Contraintes de multiplicité} \\
\midrule
%% Dépendances fonctionnelles
% Refuge
\att{emailRef}
$\rightarrow$
\att{nomRef},
\att{numTel},
\att{secGeo},
\att{dateOuv},
\att{dateFerm},
\att{nbrPlacesRepas},
\att{nbrPlacesDormir},
\att{textePrez},
\att{typePayement},
\att{prixNuite}

% ProposeRepas
\att{emailRef},
\att{typeRepas}
$\rightarrow$
\att{prixRepas}

% Formation
\att{idForm},
\att{anneeForm}
$\rightarrow$
\att{nomForm},
\att{dateDemForm},
\att{dureeForm},
\att{nbrPlacesForm},
\att{descriptForm},
\att{prixForm},
\att{nbrActivite},
\att{rangFormation}

% Lot
\att{marque},
\att{modele},
\att{anneeAchat}
$\rightarrow$
\att{nbPiecesLot},
\att{prixPertDegr},
\att{texteInfo},
\att{anneePerem}

% ArbreCategorie
\att{categorie}
$\rightarrow$
\att{categoriePere}

% AUneCategorie
\att{marque},
\att{modele},
\att{anneeAchat}
$\rightarrow$
\att{categorie}

% Membre
\att{emailMem}
$\rightarrow$
\att{pwdMem},
\att{nomMem},
\att{prenomMem},
\att{adrPostMem}

% PossedeCompte
\att{emailMem}
$\rightarrow$
\att{idUser}

% ReservationRefuge
\att{idReservRef}
$\rightarrow$
\att{dateReservRef},
\att{heureReservRef},
\att{refugeReserv},
\att{idUser},
\att{nbrNuitsReserv},
\att{nbrRepasReserv},
\att{prixTotalReserv}

% ConcerneRefuge
\att{idReservRef}
$\rightarrow$
\att{emailRef}

% ReservationFormation
\att{idReservForm}
$\rightarrow$
\att{idAdherent},
\att{idForm}

% ConcerneFormation
\att{idReservForm}
$\rightarrow$
\att{idForm},
\att{anneeForm}

% LocationMateriel
\att{idLocationMat}
$\rightarrow$
\att{idAdherent},
\att{nbPiecesReserves},
\att{dateRecup},
\att{dateRetour},
\att{nbrPiecesPerdues}

% CompteUtilisateur
\att{idUser}
$\rightarrow$
\att{coutReserv},
\att{sommeMaterielAbime},
\att{sommeRembourse}

& %% Contraintes de valeur
% Refuge
$\att{dateOuv} < \att{dateFerm}$

$\att{nbrPlacesRepas} \geq 0$

$\att{nbrPlacesDormir} \geq 0$

$\att{typePayement} \in \{ \texttt{"espèce"}, \texttt{"chèque"}, \texttt{"carte-bleue"} \}$

$\att{prixNuite} \geq 0$

% Activite
$\att{nomActivite} \in \{ \texttt{"randonnée"}, \texttt{"escalade"}, \texttt{"alpinisme"}, \texttt{"spéologie"}, \texttt{"ski de rando"}, \texttt{"cascade de glace"} \}$

% Repas
$\att{prixRepas} \geq 0$

$\att{typeRepas} \in \{ \texttt{"déjeuner"}, \texttt{"dîner"}, \texttt{"souper"}, \texttt{"casse-croûte"} \}$

% Formation
$\att{dateDemForm.annee} == \att{anneeForm}$

$\att{dureeForm} \geq 0$

$\att{nbrPlacesForm} > 0$

$\att{prixForm} \geq 0$

% Lot
$\att{nbPiecesLot} \geq 0$

$\att{prixPertDegr} \geq 0$

$\att{anneePerem} \geq 1970$

% Réservation de refuge
$\att{dateOuv} < \att{dateReservRef} < \att{dateFerm}$

$\att{nbrNuitsReserv} \geq 0$

$\att{nbrRepasReserv} \geq 0$

$\att{prixTotalReserv} > 0$

% LocationMateriel
$\att{dateRetour} < \att{dateRecup} + 14$

% ReserveRepas
$\att{nbrNuitsReserv} \neq 0 \lor \att{nbrRepasReserv} \neq 0$

& %% Contraintes de multiplicités
% ReserveRefuge
$\att{idUser} \not\twoheadrightarrow \att{idReservRef}$

% ReserveFormation
$\att{idUser} \not\twoheadrightarrow \att{idReservForm}$

% LoueMateriel
$\att{idUser} \not\twoheadrightarrow \att{idLocationMat}$

% Lot
$\att{marque}, \att{modele}, \att{anneeAchat} \not\rightarrow \att{texteInfo}$

% EstLoue
$\att{marque}, \att{modele}, \att{anneeAchat} \not\twoheadrightarrow \att{idLocationMateriel}$
\end{tabularx}
\end{adjustbox}
% \caption{Table de contraintes}
\end{table}
\section{Conception}
Tout au long du projet, on a rencontré des choix difficiles à faire face aux subtilités du problème.

\begin{itemize}
\item \textbf{Arbre de catégories}.
  Vu que dans un arbre, un noeuds fils à un noeud parent unique, on a donc choisi de modéliser cet arbre avec un table \att{Arbre} d'un seul attribut \att{categorie}.
  On met en place aussi deux associations: \att{AUneCategorie} pour lier l'entité \att{Lot} et \att{Arbre}, et \att{EstParentDe} pour lier \att{Arbre} avec lui-même pour modéliser les relations entre les noeuds.
\item \textbf{Prix d'un repas}.
  Chaque refuge propose des repas à des prix différents donc créer un table avec des attributs \att{typeRepas} et \att{prixRepas} est hors du question.
  Donc on implémente l'attribut \att{prixRepas} comme un attribut d'association \att{ProposeRepas} entre \att{Refuge} et \att{Repas}.
  Même raisonnement pour \att{nbRepasReserv}.
\item \textbf{Adhérent}.
  On a remarqué tôt dans le projet que \att{Adherent} pourrait être un sous-type de \att{Membre}.
  La question est de choisir la bonne implémentation du sous-type.
  L'implémentation d'une \enquote{Réference} nous oblige de faire une jointure avec \att{Adherent} et \att{Membre} qui n'est pas optimal.
  Une \enquote{Duplication} est donc choisie à cause de l'accès facile.
\item \textbf{Disponibilités d'une refuge à une date}.
  Un point qui pourrait être le problème plus complexe est de savoir le nombre de places disponibles dans une refuge à une date.
  Vu qu'on ne stocke pas les places disponibles à chaque jour d'ouverture à une refuge, on a choisi de stocker le nombre total de places disponibles sur la période d'ouverture d'une refuge, et puis calculer les places disponibles à une date choisie par le client.
  Ce dernier est implémenté par double itération.
  On itère d'abord sur chaque jour de la période de réservation souhaitée par le client.
  Ensuite, sur chaque jour, on itère sur toutes les réservations de refuge qui partagent le même jour et on compte le nombre de places qui reste.
  Si le nombre de places qui reste atteint 0, la période de réservation n'est pas disponible.
  Si le nombre de places qui reste est strictement positif sur toute la période, la réservation peut s'effectuer pour client.
\item \textbf{Liste d'attente}.
  Au début, on a choisi d'implémenter une liste d'attente pour des formations qui n'ont plus de places disponibles.
  Par contre, cela veut dire que chaque formation aura un table avec deux attributs: le rang sur la liste d'attente et l'identifiant d'utilisateur.
  C'est évident que ce design entraine des données redondantes.
  Donc, on a intégré le rang comme attribut dans le table des réservations de formation.
\end{itemize}

Pour la génération du diagramme entité-association, on a utilisé PlantUML qui est très pratique.
Par contre, la création des objets sans titre n'est pas possible pour les attributs d'association comme \att{prixRepas}.

\begin{figure}[H]
\includesvg[inkscapelatex=false, width=\textwidth]{schema}
\caption{Diagramme entité-association}
\end{figure}

\section{Schéma relationnel}
\subsection*{Entités simples}
\att{REFUGE}(\att{\underline{emailRef}}, \att{nomRef}, \att{numTel}, \att{secGeo}, \att{dateOuv}, \att{dateFerm}, \att{nbrPlacesRepas}, \att{nbrPlacesDormir}, \att{textePrez}, \att{typePayement}, \att{prixNuite})
\begin{itemize}
\item \textbf{Forme normale}: Non normalisé (\att{numTel} pourrait être \att{NULL})
\end{itemize}

\att{REPAS}(\att{\underline{typeRepas}})

\att{COMTPE\_UTILISATEUR}(\att{\underline{idUser}}, \att{coutReserv}, \att{sommeMaterielAbime}, \att{sommeRemboursee})
\begin{itemize}
\item \textbf{Forme normale}: 3FNBCK
\end{itemize}

\att{FORMATION}(\att{\underline{idForm}}, \att{\underline{anneeForm}}, \att{nomForm}, \att{dateDemForm}, \att{dureeForm}, \att{nbrPlacesForm}, \att{descriptForm}, \att{prixForm}, \att{nbrActivite})
\begin{itemize}
\item \textbf{Forme normale}: 3FN (car il y a $>1$ clés primaires)
\end{itemize}

\att{ACTIVITE}(\att{\underline{nomAct}})

\att{ARBRE}(\att{\underline{categorie}})

\att{MEMBRE}(\att{\underline{emailMem}}, \att{pwdMem}, \att{nomMem}, \att{prenomMem}, \att{adrPostMem}, \att{idUser})
\begin{itemize}
\item \textbf{Forme normale}: 3FNBCK
\item \att{idUser} référence \att{COMPTE\_UTILISATEUR}(\att{idUser})
\end{itemize}

\att{RESERVATION\_REFUGE}(\att{\underline{idReservRef}}, \att{dateReservRef}, \att{heureReservRef}, \att{nbrNuitsReserv}, \att{nbrRepasReserv}, \att{prixTotalReserv}, \att{emailRef}, \att{idUser})
\begin{itemize}
\item \textbf{Forme normale}: 3FNBCK
\item \att{emailRef} référence \att{REFUGE}(\att{emailRef})
\item \att{idUser} référence \att{COMPTE\_UTILISATEUR}(\att{idUser})
\end{itemize}

\att{RESERVATION\_FORMATION}(\att{\underline{idReservForm}}, \att{rangReservForm}, \att{idUser}, \att{idForm}, \att{anneeForm})
\begin{itemize}
\item \textbf{Forme normale}: 3FNBCK
\item \att{idUser} référence \att{COMPTE\_UTILISATEUR}(\att{idUser})
\item \att{idForm} référence \att{FORMATION}(\att{idForm})
\item \att{anneeForm} référence \att{FORMATION}(\att{anneeForm})
\end{itemize}

\att{LOT}(\att{\underline{marque}}, \att{\underline{modele}}, \att{\underline{anneeAchat}}, \att{nbPiecesLot}, \att{prixPerteDegr}, \att{texteInfo}, \att{anneePerem}, \att{categorie})
\begin{itemize}
\item \textbf{Forme normale}: Non normalisé (car \att{texteInfo} peut être \att{NULL})
\item \att{categorie} référence \att{ARBRE}(\att{categorie})
\end{itemize}

\att{LOCATION\_MATERIEL}(\att{\underline{idLocationMat}}, \att{nbPiecesReserv}, \att{dateRecup}, \att{dateRetour}, \att{nbPiecesPerdues}, \att{marque}, \att{modele}, \att{anneeAchat}, \att{idUser})
\begin{itemize}
\item \textbf{Forme normale}: 3FNBCK
\item \att{marque} référence \att{LOT}(\att{marque})
\item \att{modele} référence \att{LOT}(\att{modele})
\item \att{anneeAchat} référence \att{LOT}(\att{anneeAchat})
\item \att{idUser} référence \att{COMPTE\_UTILISATEUR}(\att{idUser})
\end{itemize}

\subsection*{Sous-types entités}

\att{ADHERENT}(\att{\underline{emailMem}}, \att{pwdMem}, \att{nomMem}, \att{prenomMem}, \att{adrPostMem}, \att{idAdherent})
\begin{itemize}
\item \textbf{Forme normale}: 3FNBCK
\item \att{emailMem} référence \att{MEMBRE}(\att{emailMem})
\end{itemize}

% \subsection*{Entités faibles} % (ssi une unique entité fait référence à une autre)

% \subsection*{Associations ternaires}

% \subsection*{Associations 1..1}

\subsection*{Associations 0..1}

\att{EST\_ENFANT\_DE}(\att{\underline{categorie}}, \att{categoriePere})
\begin{itemize}
\item \textbf{Forme normale}: Non normalisé (\att{categoriePere} peut être \att{NULL} si \att{categorie} est la racine de l'arbre)
\item \att{categorie} référence \att{ARBRE}(\att{categorie})
\item \att{categoriePere} référence \att{ARBRE}(\att{categorie})
\end{itemize}

\subsection*{Associations ?..*}

\att{PROPOSE\_REPAS}(\att{\underline{emailRef}}, \att{\underline{typeRepas}}, \att{prixRepas})
\begin{itemize}
\item \textbf{Forme normale}: 3FN (car il y a $>1$ clés primaires)
\item \att{emailRef} référence \att{REFUGE}(\att{emailRef})
\item \att{typeRepas} référence \att{REPAS}(\att{typeRepas})
\end{itemize}

\att{RESERVE\_REPAS}(\att{\underline{idReservRef}}, \att{\underline{typeRepas}}, \att{nbRepasRserve})
\begin{itemize}
\item \textbf{Forme normale}: 3FN (car il y a $>1$ clés primaires)
\item \att{idReservRef} référence \att{RESERVATION\_REFUGE}(\att{idReservRef})
\item \att{typeRepas} référence \att{REPAS}(\att{typeRepas})
\end{itemize}

\att{PROPOSE\_ACTIVITE\_FORM}(\att{\underline{idForm}}, \att{\underline{anneeForm}}, \att{\underline{nomAct}})
\begin{itemize}
\item \textbf{Forme normale}: 3FN (car il y a $>1$ clés primaires)
\item \att{idForm} référence \att{FORMATION}(\att{idForm})
\item \att{anneeForm} référence \att{FORMATION}(\att{anneeForm})
\item \att{nomAct} référence \att{ACTIVITE}(\att{nomAct})
\end{itemize}

\att{COMPATIBLE\_AVEC\_ACTIVITE}(\att{\underline{marque}}, \att{\underline{modele}}, \att{\underline{anneeAchat}}, \att{\underline{nomAct}})
\begin{itemize}
\item \textbf{Forme normale}: 3FN (car il y a $>1$ clés primaires)
\item \att{marque} référence \att{LOT}(\att{marque})
\item \att{modele} référence \att{LOT}(\att{modele})
\item \att{anneeAchat} référence \att{LOT}(\att{anneeAchat})
\item \att{nomAct} référence \att{ACTIVITE}(\att{nomAct})
\end{itemize}

% \subsection*{Sous-types d'associations}

\section{Bilan du projet}
% Organisation
On a vite séparé le travail du projet en tâches indépendants.
Le demonstrateur, l'interface d'utilisateur et l'interface de base de données, est implémenté principalement par Mohamed et Khalil.
Pradana a rédigé la plupart du rapport et sur la conception du modèle entité-association.
David et Kane ont implémenté les transactions en SQL2.

% Points difficiles rencontrés
\end{document}

% Local Variables:
% TeX-command-extra-options: "-shell-escape"
% End:
